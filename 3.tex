\chapter[剛体1リンクによるリンク速度最適化シミュレーション]{剛体1リンクによる\\リンク速度最適化シミュレーション}

\section{はじめに}
本研究では自作したシミュレータを用いて投擲フォームを導出し,戦略を考察したが,検討に先立ち剛体1リンクモデルでのリンク速度最適化シミュレーションを行い,シミュレータと手法の妥当性について検証した.
\section{動力学モデル}
本章の検証で用いた動力学モデルである剛体1リンクを\figref{3.1.eps}に示す.\\

\fig{3.1.eps}{width=1.0\hsize}{Rigid 1 Link}

ラグランジュの方法\cite{lagrange}を用いて導いた剛体1リンクの運動方程式は,次の通りである.\\
$l_{\mathrm{g}}$を重心までの長さとすると,リンクの重心位置$o_{\mathrm{s}}$は,

\begin{eqnarray}
  o_{\mathrm{s}} = 
              \begin{bmatrix}
              l_{\mathrm{g}}\cos\theta\\
              l_{\mathrm{g}}\sin\theta
              \end{bmatrix}
\end{eqnarray}

微分すると,

\begin{eqnarray}
  o_{\mathrm{\dot{s}}} = 
              \begin{bmatrix}
              -l_{\mathrm{g}}\sin\theta \cdot \dot{\theta}\\
              l_{\mathrm{g}}\cos\theta \cdot \dot{\theta}
              \end{bmatrix}
\end{eqnarray}

リンク重さを$m$とすると,剛体1リンクの並進運動エネルギー$E_{\mathrm{t}}$は,

\begin{eqnarray}
  E_{\mathrm{t}} 
  &=&\frac{1}{2}mo_{\mathrm{\dot{s}}}{}^\mathrm{T}\!o_{\mathrm{\dot{s}}} \nonumber \\
  &=&\frac{1}{2}m{l_{\mathrm{g}}}^2{\dot{\theta}}^2
\end{eqnarray}

同様に,リンク重心周りの慣性モーメントを$I$とすると,回転運動エネルギー$E_{\mathrm{r}}$は,

\begin{eqnarray}
  E_{\mathrm{r}} 
  =\frac{1}{2}I{\dot{\theta}}^2
\end{eqnarray}

よって,運動エネルギー$T$は,

\begin{eqnarray}
  T
  &=&E_{\mathrm{t}} + E_{\mathrm{r}} \nonumber \\
  &=&\frac{1}{2}{\dot{\theta}}^2(m{l_{\mathrm{g}}}^2 + I)
\end{eqnarray}

位置エネルギー$T$は,

\begin{eqnarray}
  U
  = mgl_{\mathrm{g}}\sin\theta
\end{eqnarray}

ここで,ラグランジュ関数$L$を

\begin{eqnarray}
  L
  =T - U
\end{eqnarray}

とすると,運動方程式は,

\begin{eqnarray}
  \equlabel{3.8}
  \frac{d}{dt}(\frac{\partial L}{\partial \dot{q_{i}}}) - \frac{\partial L}{\partial q_{i}} = f_{i} \quad (i = 1,\cdot\cdot\cdot, n)
\end{eqnarray}

ラグランジュ関数$L$を求め,\equref{3.8}に代入し粘性を考慮すると,運動方程式は,

\begin{eqnarray}
  \equlabel{3.9}
  M\ddot{\theta} + g(\theta) + b(\dot{\theta}) = \tau
\end{eqnarray}

このとき,

\begin{eqnarray}
  M = I + m{l_{\mathrm{g}}}^2
\end{eqnarray}

\begin{eqnarray}
  g(\theta) = mgl_{\mathrm{g}}\cos\theta
\end{eqnarray}

\begin{eqnarray}
  b(\dot{\theta}) = b\dot{\theta}
\end{eqnarray}

$M$は慣性項,$g(\theta)$は重力項,$b(\dot{\theta})$は粘性項である.

\section{シミュレータの作成}
剛体1リンクを強化学習するためのシミュレータをpythonで実装した.シミュレータでは導出した運動方程式をEuler法\cite{euler}により数値積分し運動学を解くことで,剛体1リンクの角度や角速度を計算した.Euler法は常微分方程式を解く手法の1つである.本章では,\equref{3.13},\equref{3.14}を用いて計算を行った.\\

\begin{eqnarray}
  \equlabel{3.13}
  \theta_{N + 1} = \theta + \dot{\theta} \cdot dt
\end{eqnarray}

\begin{eqnarray}
  \equlabel{3.14}
  \dot{\theta}_{N + 1} = \dot{\theta} + \ddot{\theta} \cdot dt
\end{eqnarray}

Euler法の時間ステップは0.001 s とした.また,各エピソードごとに出力した時系列データを基に,pythonライブラリのmatplotlibによりリンク挙動をアニメーション表示するコードを作成した.
\section{強化学習の設定}
本章の強化学習を行うにあたって,状態,行動,報酬,その他を設定した.
\subsection{状態}
Q学習はQ値をQテーブルと呼ばれる状態と行動で表される表に格納するため,連続値を離散化する必要がある.
状態変数は2つとし,リンクの角度$\theta$と角速度$\dot{\theta}$である.
角度の範囲は$-180$ deg $\leq \theta \leq$ $180$ deg,角速度の範囲は$-2.0$ m/s $\leq \dot{\theta} \leq$ $2.0$ m/sとした.
また,分割数は各状態4分割であり,全ての状態を$4^{2}$=16通りで表すことができる.
\subsection{行動}
行動は,全3通りに設定した.回転ジョイントにかかるトルクの選択肢を$-5.0$ N$\cdot$m,$0$,$+5.0$ N$\cdot$mとし,いずれかを$\epsilon$-greedy法に基づき選択した.
\subsection{報酬}
報酬の設計について,\equref{3.15}に示す.リンクの角速度を報酬とすることで,リンクの角速度を大きくする設定とした.

\begin{eqnarray}
  \equlabel{3.15}
  reward = \dot{\theta}
\end{eqnarray}

\subsection{その他}
本検証において,学習率$\alpha$=0.1,割引率$\gamma$=0.9とした.$\epsilon$は\equref{3.16}であり,エピソードが進むにつれてランダム値を選択する確率を小さくする設定とした.

\begin{eqnarray}
  \equlabel{3.16}
  \epsilon = 0.3 \times 0.99^{(\mathrm{episode} + 1)}
\end{eqnarray}

また,エピソード数を1000,時間ステップ$dt$を0.01,1エピソードあたりのステップ数を3000とし,30 s 間の剛体1リンクのリンク速度最適化シミュレーションを行った.
\section{リンク速度最適化シミュレーション}
\subsection{パラメータの設定}
パラメータの設定について,\tabref{3.1}に示す.$b$は粘性係数である.
\begin{table}[tb]
  \begin{center}
    \caption{Rigid 1 link parameters}
    \tablabel{3.1}
    \begin{tabular}{c|c|c}
      \hline
      Parameters & Unit & Values \\
      \hline
      $m$ & kg & 1.0 or 5.0 or 10.0\\
      $l$ & m & 1.00 \\
      $b$ &  & 0.01 \\
      \hline
    \end{tabular}
  \end{center}
\end{table}
リンクの重さは,1.0 kg,5.0 kg,10.0 kg の3種類とした.
\subsection{結果・考察}
初期条件は,$\theta$ = $-90$ deg ,$\dot{\theta}$ = 0 とした際の,リンク重さに応じたリンク速度最適化シミュレーションを行った結果について述べる.\figref{3.2.eps}はリンク重さ1.0 kg,5.0 kg,10.0 kg の際の報酬の遷移である.また,\figref{3.3.eps}はそれぞれのリンク重さの際の30 s 間のリンク速度の時系列を比較したグラフである.比較の結果,リンク重さによってリンク速度の推移に違いがみられた.特に,$m$=10.0 kg の際にトルクの入力は30秒間 $+ 5$ N$\cdot$m が選択され続けていたが,振動開始直後にややリンク速度が負となる瞬間がある.これはトルク不足の影響が考えられる.リンクを回転させる際,最もトルクが必要な姿勢はリンクを垂直下向きの位置から振り上げるタイミングである.本検証は剛体1リンクで自由度が1のため,他のリンクや関節による干渉を受けずリンクが回転し続けた.リンクや自由度が多くなると,他のリンクや関節による影響でリンク挙動に違いが生じると考えられる.\\
 本検証により,自作したシミュレータと手法の有用性を確認した.次章以降では,腕に見立てた剛体2リンクを動力学モデルとし,投擲フォームの導出・比較,戦略の考察を行う.
\fig{3.2.eps}{width=.70\hsize}{Reward progress in Rigid 1 Link(1.0 kg,5.0 kg,10.0 kg)}
\fig{3.3.eps}{width=1.0\hsize}{Transition in angular velocity of Rigid 1 Link(1.0 kg,5.0 kg,10.0 kg)}

