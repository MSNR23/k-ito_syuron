\chapter[結論および今後の展望]{結論および今後の展望}

\section{結論}
本研究では投擲物の重さや身体のパラメータに応じた投擲フォームを導出・比較することで,パラメータに応じた投擲フォーム戦略の違いを考察した.まず,剛体1リンクモデルにより強化学習を用いたリンク速度最適化シミュレーションを行うことで,自作した強化学習シミュレータと手法の有用性を確認した.その後,腕に見立てた2次元剛体2リンクモデルへと拡張し,投擲物の重さや腕の長さに応じた投擲フォーム戦略の考察を行い,パラメータによって投擲フォーム戦略に違いが生じることを確認した.その後,より人間に近い3次元の腕モデルへと拡張し,投擲物の重さや腕の長さに応じた投擲フォーム戦略について考察を行った.投擲物の重さに応じた投擲フォーム戦略の比較の結果,腕の押し出し度合いによる戦略の違いがみられた.また,腕の長さに応じた投擲フォーム戦略の比較の結果,投擲開始から慣性モーメントの影響により挙動に違いが生じたが,ともに運動連鎖による肩関節の回転を重要視した戦略がみられた.

\section{今後の展望}
今後の展望として,深層強化学習を用いたトルクの連続値入力による学習,変更する投擲物の重さや身体のパラメータの種類の増加がある.また,本研究では腕による投擲フォーム戦略の考察を行ったが,2次元から3次元の拡張で新たに追加された要素によって新たな戦略もみられた.そのため,全身モデルでの学習による投擲フォーム戦略の考察により,全身の運動連鎖の傾向等,より実際の人間に近い投擲フォーム戦略の考察が可能であると考えられる.加えてさまざまなスポーツに応じたルール制約を設けた学習により,実際の競技や個人に応じた投擲フォーム戦略がみられると考えられる.
