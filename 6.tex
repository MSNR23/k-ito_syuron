\chapter[結論および今後の展望]{結論および今後の展望}

\section{結論}
本研究では投擲物の重さや身体のパラメータに応じた投擲フォームを導出・比較することで,パラメータに応じた投擲フォーム戦略の違いを考察した.まず,剛体1リンクモデルにより強化学習を用いたリンク速度最適化シミュレーションを行うことで,自作した強化学習シミュレータと手法の有用性を確認した.その後,腕に見立てた2次元剛体2リンクモデルへと拡張し,投擲物の重さや腕の長さに応じた投擲フォーム戦略の考察を行い,パラメータによって投擲フォーム戦略に違いが生じることを確認した.その後,より人間に近い3次元の腕モデルへと拡張し,投擲物の重さや腕の長さに応じた投擲フォーム戦略について考察を行った.投擲物の重さに応じた遠投をするための投擲フォーム戦略の比較の結果,腕の押し出し度合いによる戦略の違いがみられた.腕の長さに応じた遠投をするための投擲フォーム戦略の比較の結果,投擲開始から慣性モーメントの影響により挙動に違いが生じたが,ともに運動連鎖による肩関節の回転を重要視した戦略がみられた.また,本研究の手法は投擲フォーム戦略を考察することにおいて有用であることが示された.

\section{今後の展望}
今後の展望として,深層強化学習を用いたトルクの連続値入力による学習,変更する投擲物の重さや身体のパラメータの種類の増加がある.また,本研究では腕による投擲フォーム戦略の考察を行ったが,2次元から3次元の拡張で追加された要素によって新たな戦略もみられた.そのため,全身モデルでの学習による投擲フォーム戦略の考察により,全身の運動連鎖の傾向等,より実際の人間に近い投擲フォーム戦略の考察が可能であると考えられる.加えてさまざまなスポーツに応じたルール制約を設けた学習により,実際の競技や個人に応じた投擲フォーム戦略がみられると考えられる.
