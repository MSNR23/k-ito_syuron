\chapter[2次元剛体2リンクによるパラメータに応じた投擲フォーム戦略の考察]{2次元剛体2リンクによるパラメータに応じた投擲フォーム戦略の考察}

\section{はじめに}
本章では、2次元モデルによる投擲物の重さに応じた投擲フォームの比較により、考察した投擲フォーム戦略について述べる。同様に腕の長さに応じた投擲フォームの比較により、考察した投擲フォーム戦略について述べる。
\section{動力学モデル}
本章の検証で用いた動力学モデルを\figref{4.1.eps}
に示す。人間の腕を肩関節1自由度と肘関節1自由度の計2自由度から構成される2次元剛体2リンクとしてモデル化した。
\fig{4.1.eps}{width=1.0\hsize}{剛体2リンク}
ラグランジュの方法を用いて導いた剛体2リンクの運動方程式は、次の通りである。\\
各リンクの重心位置$o_{s_{1}}$,$o_{s_{1}}$は、

\begin{eqnarray}
  o_{s_{1}} = 
              \begin{bmatrix}
              l_{g1}\cos\theta_{1}\\
              l_{g1}\sin\theta_{1}
              \end{bmatrix}
\end{eqnarray}

\begin{eqnarray}
  o_{s_{2}} = 
              \begin{bmatrix}
              l_{1}\cos\theta_{1} + l_{g2}\cos(\theta_{1} + \theta_{2})\\
              l_{1}\sin\theta_{1} + l_{g2}\sin(\theta_{1} + \theta_{2})
              \end{bmatrix}
\end{eqnarray}

微分すると、

\begin{eqnarray}
  o_{\dot{s}_{1}} = 
              \begin{bmatrix}
              -l_{g1}\sin\theta_{1}\dot{\theta}_{1}\\
              l_{g1}\cos\theta_{1}\dot{\theta}_{1}
              \end{bmatrix}
\end{eqnarray}
\begin{eqnarray}
  o_{\dot{s}_{2}} = 
              \begin{bmatrix}
                -l_{1}\sin\theta_{1}\dot{\theta}_{1} - l_{g2}\sin(\theta_{1} + \theta_{2})(\dot{\theta}_{1} + \dot{\theta}_{2})\\
                l_{1}\cos\theta_{1}\dot{\theta}_{1} + l_{g2}\cos(\theta_{1} + \theta_{2})(\dot{\theta}_{1} + \dot{\theta}_{2})
              \end{bmatrix}
\end{eqnarray}

剛体2リンク全体の並進運動エネルギー$E_{t}$は、各リンクの並進運動エネルギーの合計とすると、

\begin{eqnarray}
  E_{t}
  &=&\frac{1}{2}m_{1}o_{\dot{s}_{1}}{}^T\!o_{\dot{s}_{1}} + \frac{1}{2}m_{2}o_{\dot{s}_{2}}{}^T\!o_{\dot{s}_{2}} \nonumber \\
  &=&\frac{1}{2}m_{1}{l_{g1}}^2{\dot{\theta}_{1}}^2 + \frac{1}{2}m_{2}({{l_{1}}^2}{\dot{\theta}_{1}}^2 + {l_{g2}}^2(\dot{\theta}_{1} + \dot{\theta}_{2})^2 + 2{l_{1}}{l_{g2}}\cos({\dot{\theta}_{1}}^2 + \dot{\theta}_{1}\dot{\theta}_{2}))
\end{eqnarray}

同様に、回転運動エネルギー$E_{r}$は、

\begin{eqnarray}
  E_{r}
  =\frac{1}{2}I_{1}{\dot{\theta}_{1}}^2 + \frac{1}{2}I_{2}(\dot{\theta}_{1} + \dot{\theta}_{2})^2
\end{eqnarray}

全体の運動エネルギー$E_{k}$は、

\begin{eqnarray}
  E_{k}
  =E_{t} + E_{r}
\end{eqnarray}

重力によって生じる位置エネルギーについて、重力方向をg = ${[0 \quad -g]}^T$とすると、上腕リンクの位置エネルギー$E_{p1}$は、

\begin{eqnarray}
  E_{p1}
  =m_{1}gl_{g1}\sin\theta_{1}
\end{eqnarray}

前腕リンクの位置エネルギー$E_{p2}$は、

\begin{eqnarray}
  E_{p1}
  =m_{2}g(l_{1}\sin\theta_{1} + l_{g2}\sin(\theta_{1} + \theta_{2}))
\end{eqnarray}

全体の位置エネルギー$K_{p}$は、

\begin{eqnarray}
  E_{p}
  =E_{p1} + E_{p2}
\end{eqnarray}

ここで、ラグランジュ関数Lを

\begin{eqnarray}
  L
  =E_{k} - E_{p}
\end{eqnarray}

とする。ラグランジュ関数Lを求め、\equref{3.8}に代入すると、$\theta_{1}$についての運動方程式

\begin{eqnarray}
  &&(m_{1}{l_{g1}}^2 + I_{1} + m_{2}({l_{1}}^2 + {l_{g2}}^2)
  + 2l_{1}l_{g2}\cos\theta_{2})\ddot{\theta_{1}} \nonumber \\
  &+& (m_{2}({l_{g2}}^2 + l_{1}l_{g2}\cos\theta_{2}) + I_{2})\ddot{\theta_{2}} \nonumber \\
  &-& m_{2}l_{1}l_{g2}\sin\theta_{2}(2\dot{\theta_{1}}\dot{\theta_{2}} + {\dot{\theta_{2}}}^2)  \nonumber \\
  &+& m_{1}gl_{g1}\cos\theta_{1} + m_{2}g(l_{1}\cos\theta_{1} + l_{g2}\cos(\theta_{1} + \theta_{2})) = \tau_{1}
\end{eqnarray}

$\theta_{2}$についての運動方程式

\begin{eqnarray}
  &&(m_{2}({l_{g2}}^2 + l_{1}l_{g2}\cos\theta_{2}) + I_{2})\ddot{\theta_{1}} + (m_{2}{l_{g2}}^2 + I_{2})\ddot{\theta_{2}} \nonumber \\
  &+& m_{2}l_{1}l_{g2}\sin\theta_{2}{\dot{\theta_{1}}}^2 + m_{2}gl_{g2}\cos(\theta_{1} + \theta_{2}) = \tau_{2}
\end{eqnarray}

を得る。これらを整理すると以下の式となる。

\begin{eqnarray}
  M(\theta)\ddot{\theta} + h(\theta, \dot{\theta}) + g(\theta) + b(\dot{\theta}) = \tau
\end{eqnarray}

このとき、$\theta$=${[\theta_{1} \quad \theta_{2}]}^T$、$\tau$=${[\tau_{1} \quad \tau_{2}]}^T$であり、粘性を考慮すると、

\begin{align}
  M(\theta)= 
            \begin{bmatrix}
              m_{1}{l_{g1}}^2 + I_{1} + m_{2}({l_{1}}^2 + {l_{g2}}^2 + 2l_{2}l{g2}\cos\theta{2}) + I_{2} & m_{2}({l_{g2}}^2 + l_{1}l{g2}\cos\theta_{2}) + I_{2} \\
              m_{2}({l_{g2}}^2 + l_{1}l{g2}\cos\theta_{2}) + I_{2} & m_{2}{l_{g2}}^2 + I_{2}
            \end{bmatrix}
\end{align}

\begin{eqnarray}
  h(\theta, \dot{\theta})=
            \begin{bmatrix}
              -m_{2}l_{1}l_{g2}\sin\theta_{2}(2\dot{\theta_{1}}\dot{\theta_{2}} + {\dot{\theta_{2}}}^2) \\
              m_{2}l_{1}l_{g2}\sin\theta_{2}{\dot{\theta_{1}}}^2
            \end{bmatrix}
\end{eqnarray}

\begin{eqnarray}
  g(\theta)=
            \begin{bmatrix}
              m_{1}gl_{g1}\cos\theta_{1} + m_{2}g(l_{1}\cos\theta_{1} + l_{g2}\cos(\theta_{1} + \theta_{2})) \\
              m_{2}gl_{g2}\cos(\theta_{1} + \theta_{2})
            \end{bmatrix}
\end{eqnarray}

\begin{eqnarray}
  b(\dot{\theta})=
            \begin{bmatrix}
              b_{1}\dot{\theta_{1}} \\
              b_{2}\dot{\theta_{2}}
            \end{bmatrix}
\end{eqnarray}

M($\theta$)は慣性行列、h($\theta$, $\dot{\theta}$)はコリオリ力の項、g($\theta$)は重力項、b($\dot{\theta}$)は粘性項である。
\section{シミュレータの作成}
剛体1リンクを強化学習するためのQ学習シミュレータをpythonで実装した。シミュレータでは導出した運動方程式をEuler法により数値積分し運動学を解くことで、剛体1リンクの角度や角速度を計算する。Euler法は上微分方程式を解く手法の1つである。
\section{強化学習の設定}
\subsection{状態}
状態変数は4つとし、肩関節の角度$\theta_{1}$、角速度$\dot{\theta}_{1}$,肘関節の角度$\theta_{2}$、角速度$\dot{\theta}_{2}$とした。\\
各関節の角度の可動範囲は、-180 deg $\le$ $\theta_{1}$ $\le$ 180 deg、0 $\le$ $\theta_{2}$ $\le$ 150 degとした。
上腕リンク、すなわち肩は本来3自由度でありピッチ軸周りの可動範囲は設定した可動範囲より小さい。しかし、ロール軸、ヨー軸の可動範囲の影響により肩が360度回転しているように見える。よって、本章では2次元でのシミュレーションのため上記の可動範囲とした。
また、各リンクの角速度については、-10.0 m/s $\le$ $\dot{\theta}_{1}$ $\le$ 10.0 m/s、-10.0 m/s $\le$ $\dot{\theta}_{2}$ $\le$ 10.0 m/sとした。
分割数は各状態4分割であり、全ての状態を$4^{4}$=256通りで表すことができる。
\subsection{行動}
行動は、全9通りに設定した。肩関節にかかるトルクを正、0、負の3通り、同様に肘関節にかかるトルクも正、0、負の3通りとした。
これにより、Qテーブルは256$\times$9=2304通りで表すことができる。
\subsection{報酬}
報酬は、リリース時のa$\theta_{v}$と手先速度$v_{2}$の2つの要素から設定した。リリースのa$\theta_{v}$について、で計算した。
\begin{eqnarray}
  \equlabel{2.20}
  \theta_{v}=\theta_{1}+\theta_{2}-90
\end{eqnarray}
\equref{2.20}において、$\theta_{v}$、$\theta_{1}$、$\theta_{2}$の単位はdegである。
また、条件は以下の通りとした。\\
0 $\le$ $\theta_{v}$ $\le$ 45 degのとき、$\frac{\theta_{v}}{45}$\\
45 deg $\le$ $\theta_{v}$ $\le$ 90 degのとき、$\frac{90-\theta_{v}}{45}$\\
90 deg $\le$ $\theta_{v}$ $\le$ 135 degのとき、$\frac{\theta_{v}-90}{45}$\\
135 deg $\le$ $\theta_{v}$ $\le$ 180 degのとき、$\frac{180-\theta_{v}}{45}$\\
$\theta_{v}$ $\le$ 0、180 $\le$ $\theta_{v}$のとき、0\\

この条件により、$\theta_{v}$が45 degまたは135 degのときに報酬最大となる。これは、投擲物の遠投時に最も有効な角度である。
これらの角度からリリース時の角度が離れるにつれて報酬が小さくなっていく。
よって投擲方向は指定せずにより投擲物の飛距離を大きくするためのリリースポイントの報酬を大きくすることができる。\\
手先速度$v_{2}$について、

\begin{eqnarray}
  \equlabel{2.21}
  v_{x2}^{2}=-l_{1}\sin\theta_{1}\cdot\dot{\theta}_{1}-l_2\sin(\theta_{1}+\theta_{2})\cdot(\dot{\theta}_{1}+\dot{\theta}_{2})
\end{eqnarray}

\begin{eqnarray}
  \equlabel{2.22}
  v_{y2}^{2}=-l_{1}\cos\theta_{1}\cdot\dot{\theta}_{1}+l_2\cos(\theta_{1}+\theta_{2})\cdot(\dot{\theta}_{1}+\dot{\theta}_{2})
\end{eqnarray}

\equref{2.21}、\equref{2.22}より、

\begin{eqnarray}
  \equlabel{2.23}
  v_{2}=\sqrt{v_{x2}^{2}+v_{y2}^{2}}
\end{eqnarray}

で求めた。\\

以上の2つの要素より、報酬は$\theta_{v} \times v_{2}$と設定した。この設定により、遠投を行うための投擲フォームを獲得することができる。
また、リリース時の投射角と手先速度の積とすることで、どちらの要素にも報酬に対する重みを付与することができる。\\
さらに効率的に投擲フォームを獲得するために、罰則として累積消費エネルギーを採用し、\equref{2.26}で計算した。

\begin{eqnarray}
  \equlabel{2.24}
  CE_{1}&=&\int \tau_{1}(t)\cdot\dot{\theta}_{1} dt\\
        &\approx&|\tau_{1}(t)\cdot\dot{\theta}_{1}|\cdot\Delta t
\end{eqnarray}

\begin{eqnarray}
  \equlabel{2.25}
  CE_{2}&=&\int \tau_{2}(t)\cdot\dot{\theta}_{2} dt\\
        &\approx&|\tau_{2}(t)\cdot\dot{\theta}_{2}|\cdot\Delta t
\end{eqnarray}

\equref{2.24}、\equref{2.25}より、

\begin{eqnarray}
  \equlabel{2.26}
  CE=CE_{1}+CE_{2}
\end{eqnarray}

$\tau_{1}$は肩関節に与えるトルク、$\tau_{2}$は肘関節に与えるトルクである。これは、1エピソード内において、ステップごとに消費するエネルギーを計算し、それを加算したものである。
累積消費エネルギーはステップを重ねるごとに大きくなるため、1エピソードの内、最終ステップが最も大きくなる。
シミュレーション上では上腕リンクを回し続けるほど手先速度は大きくなると考えられるが、実際の投擲動作では人間の疲労を考慮する必要がある。
そのため、累積消費エネルギーを罰則として採用することは、疲労を再現することを期待できる。\\
以上より、報酬の設計は、

\begin{eqnarray}
  \equlabel{2.27}
  Reward = $\theta_{v} \times v_{2}$ + CE
\end{eqnarray}

\subsection{その他}
本章では、\equref{Q}における学習率$\alpha$を0.1、割引率$\gamma$を0.9に設定した。これは将来の報酬を優先的に考慮した学習を期待するためである。
1タイムステップは0.005 sであり、1エピソード内のステップ数は2000、すなわち、10 sとした。エピソード数に関しては、学習が収束するまで制限なく行うものとした。
\section{投擲物の重さに応じた最適投擲フォームの比較}
\subsection{シミュレーション設定}
本検証で用いたパラメータを表\tabref{4.1}に示す。身体のサイズは、身長1.72 m、体重70 kgの人間の各部位のサイズとした。
投擲物は、0.14 kg(野球の硬式球)と7.24 kg(砲丸)とした。
\begin{table}[tb]
  \tablabel{4.1}
  \begin{center}
    \caption{Parameters which are used for 2-link simulation}
    \begin{tabular}{c|c|c}
      \hline
      Parameters & Unit & Values \\
      \hline
      $l_{1}$ & m & 0.32 \\
      $l_{2}$ & m & 0.44 \\
      $l_{f}$ & m & 0.25 \\
      $l_{h}$ & m & 0.19 \\
      $m_{f}$ & kg & 1.12 \\
      $m_{h}$ & kg & 0.42 \\
      $m_{o}$ & kg & 0.14 or 7.24 \\
      $m_{1}$ & kg & 1.96 \\
      $m_{2}$ & kg & 1.68 or 8.78 \\
      $b_{1}$ &  & 0.05 \\
      $b_{2}$ &  & 0.01 \\
      \hline
    \end{tabular}
  \end{center}
\end{table}
重力加速度gを9.8 $m/s^{2}$とする。また、本章では、前腕と手と投擲物の要素を合わせて前腕リンクとしてみなすため、3つの要素を考慮して設定を行った。
前腕の要素として、$l_{f}$を前腕長さ、$m_{f}$を前腕重さ、$l_{h}$を手長さ、$m_{h}$を手重さとした。\\
上腕リンクの重心までの長さ$l_{g1}$、前腕リンクの重心までの長さ$l_{g2}$を、\equref{2.26}、\equref{2.27}により計算した。
\begin{eqnarray}
  \equlabel{2.26}
  l_{g1}=\frac{1}{2}l_{1}
\end{eqnarray}

\begin{eqnarray}
  \equlabel{2.27}
  l_{g2}=\frac{1}{2}m_{f}l_{f}+m_{h}(l_{f}+\frac{1}{2}l_{h}+m_{o}l_{2})/m_{2}
\end{eqnarray}

上腕リンクの慣性モーメント$I_{1}$と前腕リンクの慣性モーメント$I_{2}$を\equref{2.28}、\equref{2.29}により計算した。
\begin{eqnarray}
  \equlabel{2.28}
  I_{1}=\frac{1}{12}m_{1}l_{1}^{2}
\end{eqnarray}
\begin{eqnarray}
  \equlabel{2.29}
  I_{2}=\frac{1}{12}m_{f}l_{f}^{2}+m_{f}l_{g1}^{2}+\frac{1}{12}m_{h}l_{h}^{2}+m_{h}l_{g2}^{2}+m_{o}l_{2}^{2}
\end{eqnarray}

各ステップで与えるトルクについて、肩関節に与えるトルク$\tau_{1}$を\equref{2.30}、肘関節に与えるトルク$\tau_{2}$を\equref{2.31}を基に計算した。

\begin{eqnarray}
  \equlabel{2.30}
  \tau_{1} = \tau_{2} + m_{1}gl_{1}\cos(\theta_{1} + \theta_{2})
\end{eqnarray}

\begin{eqnarray}
  \equlabel{2.31}
  \tau_{2} = (m_{2} + m_{o})gl_{2}\cos(\theta_{1} + \theta_{2})
\end{eqnarray}

腕を360 deg 回転させるとき、重力の影響で$\theta_{1}$、$\theta_{2}$が0から90 deg に振り上げる際に最もトルクが必要である。そのため、$\theta_{1}$ = $\theta_{2}$ = 0の際に各関節にかかるトルクを計算し、その計算結果を基に計算した。また、肘関節の可動範囲は0 $\le$ $\theta_{2}$ $\le$ 150 deg であるが、可動範囲の端である$\theta_{2}$ = 0 または $\theta_{2}$ = 150 degでは、可動範囲内に戻ろうとする力が働く。これを考慮し、可動範囲の端では肘関節が可動範囲内に戻る方向に20 Nm のトルクを加算する設定とした。

野球の硬式球の重さの投擲物と砲丸重さの投擲物のときのそれぞれの\equref{2.26}から\equref{2.31}の計算結果を表\tabref{4.2}に示す。
\begin{table}[tb]
  \tablabel{4.2}
  \begin{center}
    \caption{Calculation Results of $l_{g1}$,$l_{g2}$,$I_{1}$,$I_{2}$}
    \begin{tabular}{c|c|c|c}
      \hline
      Parameters & unit & Values of baseball & Values of cannonball \\
      \hline
      $l_{g1}$ & m & 0.16 & 0.16 \\
      $l_{g2}$ & m & 0.21 & 0.40 \\
      $I_{1}$ & kg $\cdot$ ${m}^2$ & 0.017 & 0.017 \\
      $I_{2}$ & kg $\cdot$ ${m}^2$ & 0.081 & 1.50 \\
      $\tau_{1}$ & Nm & -16, 0, 16 & -47, 0, 47 \\
      $\tau_{2}$ & Nm & -10, 0, 10 & -41, 0, 41 \\
      \hline
    \end{tabular}
  \end{center}
\end{table}

投擲物に応じた投擲フォームの強化学習を行った際の報酬の遷移をに示す。硬式野球の重さの投擲物を投じる投擲フォームは1250エピソード、砲丸の重さの投擲物を投じる投擲フォームは4000エピソードを採用した。
\fig{4.2.eps}{width=1.0\hsize}{報酬の遷移}

\subsection{結果・考察}
$\theta_{1}$、$\theta_{2}$、$\dot{\theta}_{1}$、$\dot{\theta}_{2}$の初期状態はいずれも0としたときの投擲物に応じた投擲フォーム戦略の結果・考察を以下に示す。
\figref{4.2.eps}は強化学習により最適化した、野球の硬式球の重さの投擲物を遠投するための投擲フォーム、\figref{4.2.eps}は強化学習により最適化した、砲丸の重さの投擲物を遠投するための投擲フォームである。投擲フォームは投擲開始からリリースまでとし、リリース時の姿勢は黒の太枠で囲まれたものである。図中の青リンク、橙リンクがそれぞれ上腕リンク、前腕リンクに対応し、赤先が手先の軌跡を示す。2つの投擲フォームの比較により、リリース時の肩関節の角度と軌跡の径の大きさに違いが見られた。野球の硬式球の重さの投擲物を遠投するための投擲フォームは肩関節の角度が約157 deg から屈曲した肘が伸展し、リリース時の上腕の角度は約46 deg であった。一方、砲丸の重さの投擲物を遠投するための投擲フォームは肩関節の角度が約66 deg から屈曲した肘が伸展し、リリース時の上腕の角度は約5 deg であった。\\
\figref{4.1.eps}は投擲開始からリリースまでの肩関節・肘関節の角度、角速度、角加速度、トルクの推移である。\figref{4.2.eps}(左)は野球の硬式球の重さの投擲物を遠投するための投擲フォームの各時系列、\figref{4.2.eps}(右)は野球の硬式球の重さの投擲物を遠投するための投擲フォームの各時系列である。\\
トルクの時系列より、野球の硬式球の重さの投擲物を遠投するための投擲フォームは、肩関節と肘関節の運動連鎖を活かした投擲フォームの傾向が見られた。投擲開始からまず肩関節のトルクによって腕を回転させる。その後、約0.13 s で肩トルクが腕の回転とは逆方向にトルクが生じ、肘のトルクが腕の回転と同方向にトルクが生じリリースを迎える。これは肩関節のエネルギーを肘関節に伝達することでより手先方向のエネルギーを大きくしていると考えられる。また、肩関節と肘関節の角速度の推移より、トルクの入力に運動連鎖の傾向が見られる約0.13 s から肩関節の角速度が減少し肘関節の角速度が増加する。リリース時には肩関節と肘関節の角速度が逆転していることから最後は前腕の振りを如何に大きくするのかが重要であると考えられる。一方、砲丸の重さの投擲物を遠投するための投擲フォームは、肩関節による影響が大きい投擲フォームの傾向が見られた。投擲開始から肩関節のトルクによって腕を回転させる傾向は野球の硬式球の重さの投擲物の際と同様であるが、肩関節のトルクが0となる瞬間が数回生じ、また肘関節のトルクは肘を屈曲させる方向に生じ続けている。これは、肩関節のトルクによる腕の回転はペナルティである累積消費エネルギーへの影響が大きいためそれを抑制するためであると考えられる。また、肘関節のトルクによって肘を屈曲させ続ける傾向にある理由として、モーメントアームを小さくすることで肩の回転に消費するエネルギーを抑制するためであると考えられる。加えて肩関節と肘関節の角速度の推移より、肘関節の角速度が正となるのが遅く、リリースの瞬間も肩関節の角速度が肘関節の角速度より大きい。すなわち、肩関節の影響が野球の硬式球の重さの投擲物を遠投する際よりも大きいと考えられる。

\fig{4.2.eps}{width=.50\hsize}{剛体2リンク}



\section{腕の長さに応じた最適投擲フォームの比較}
\subsection{シミュレーション設定}
本検証で用いたパラメータを表\tabref{4.3}に示す。身体のサイズは、身長1.72 m または 1.90 m 、体重70 kgの人間の各部位のサイズとした。
投擲物は、0.14 kg(野球の硬式球)とした。
\begin{table}[tb]
  \tablabel{4.3}
  \begin{center}
    \caption{Parameters which are used for 2-link simulation}
    \begin{tabular}{c|c|c}
      \hline
      Parameters & Unit & Values \\
      \hline
      $l_{1}$ & m & 0.32 or 0.35 \\
      $l_{2}$ & m & 0.44 or 0.48 \\
      $l_{f}$ & m & 0.25 or 0.28 \\
      $l_{h}$ & m & 0.19 or 0.20 \\
      $m_{f}$ & kg & 1.12 \\
      $m_{h}$ & kg & 0.42 \\
      $m_{o}$ & kg & 0.14 \\
      $m_{1}$ & kg & 1.96 \\
      $m_{2}$ & kg & 1.68 \\
      $b_{1}$ &  & 0.05 \\
      $b_{2}$ &  & 0.01 \\
      \hline
    \end{tabular}
  \end{center}
\end{table}


身長1.72 m の人間に基づいた腕の長さでの投擲のときと身長1.90 m の人間に基づいた腕の長さでの投擲のときのそれぞれの\equref{2.26}から\equref{2.31}の計算結果を表\tabref{4.4}に示す。重力加速度、可動範囲の端での肩関節と肘関節に加算するトルクは、4.4.1節と同様の設定とした。
\begin{table}[tb]
  \tablabel{4.4}
  \begin{center}
    \caption{Calculation Results of $l_{g1}$,$l_{g2}$,$I_{1}$,$I_{2}$}
    \begin{tabular}{l|c|c|r}
      \hline
      Parameters & unit & Values of 1.72 m & Values of 1.90 m \\
      \hline
      $l_{g1}$ & m & 0.16 & 0.18 \\
      $l_{g2}$ & m & 0.21 & 0.23 \\
      $I_{1}$ & kg $\cdot$ ${m}^2$ & 0.017 & 0.018 \\
      $I_{2}$ & kg $\cdot$ ${m}^2$ & 0.020 & 0.099 \\
      $\tau_{1}$ & Nm & -16, 0, 16 & -18, 0, 18 \\
      $\tau_{2}$ & Nm & -10, 0, 10 & -11, 0, 11 \\
      \hline
    \end{tabular}
  \end{center}
\end{table}

投擲物に応じた投擲フォームの強化学習を行った際の報酬の遷移を\figref{4.4.eps}に示す。身長1.72 m の人間に基づいた腕の長さによる投擲フォームは1250エピソード、身長1.90 m の人間に基づいた腕の長さによる投擲フォームは2070エピソードを採用した。
\fig{4.4.eps}{width=1.0\hsize}{報酬の遷移}

\subsection{結果・考察}

