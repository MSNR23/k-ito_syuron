\chapter[3次元剛体2リンクによるパラメータに応じた投擲フォーム戦略の考察]{3次元剛体2リンクによるパラメータに応じた投擲フォーム戦略の考察}

\section{はじめに}
前章では2次元モデルでの投擲フォーム戦略の考察を行ったが、3次元に拡張することにより、より人間に近いモデルでの投擲フォーム戦略の考察が可能となると考えられる。そこで本章では、3次元モデルによる投擲物の重さに応じた投擲フォームの比較により、考察した投擲フォーム戦略について述べる。
\section{動力学モデル}
本章で用いた動力学モデルを\figref{5.1.jpg}に示す。本章では、\figref{5.2.jpg}で示す物理エンジンMuJoCoに標準搭載されているhumanoidモデル「Unitree G1」を改変し、人間の腕を肩関節3自由度と肘関節1自由度の計4自由度から構成される3次元剛体2リンクとしてモデル化した。なお、\figref{5.1.jpg}において、肩関節から肘関節までを上腕リンク、肘関節から手先までを前腕リンクとし、手首や指の自由度は0とした。また、体幹リンクも自由度0とし、世界座標に固定した。
\fig{5.1.jpg}{width=0.5\hsize}{動力学モデル}
\fig{5.2.jpg}{width=0.5\hsize}{Unitree G1}
\section{シミュレータの作成}
3次元剛体2リンクを強化学習するためのQ学習シミュレータをpythonで実装した。運動方程式は、MuJoCoで内部的に解いた。runge-kutta法により数値積分し運動学を解くことで、3次元剛体2リンクの角度や角速度を計算する。
\section{可動範囲}
\section{強化学習の設定}
\subsection{状態}
状態変数は8つとし、肩関節ピッチ軸まわりの角度$\theta_{p}$、角速度$\dot{\theta}_{p}$,肩関節ロール軸まわりの角度$\theta_{r}$、角速度$\dot{\theta}_{r}$,肩関節ヨー軸まわりの角度$\theta_{y}$、角速度$\dot{\theta}_{y}$,肘関節の角度$\theta_{e}$、角速度$\dot{\theta}_{e}$とした。\\
各関節の角度については、5.4章の通りである。
また、各関節の角速度については、-10.0 m/s $\le$ $\dot{\theta}_{1}$ $\le$ 10.0 m/s、-10.0 m/s $\le$ $\dot{\theta}_{2}$ $\le$ 10.0 m/sとした。
分割数は各角度が5分割、各角速度が2分割であり、全ての状態を$5^{4}\times 2^{4}$=10000通りで表すことができる。
\subsection{行動}
行動は、全625通りに設定した。肩関節にかかるトルクを正2通り、0、負2通りの計5通り、同様に肘関節にかかるトルクも正2通り、0、負2通りの計5通りとした。
これにより、Qテーブルは256$\times$9=2304通りで表すことができる。
\subsection{報酬}
\subsection{その他}
\section{投擲物の重さに応じた最適投擲フォームの比較}
\subsection{シミュレーション設定}
\section{結果・考察}
