\chapter[3次元剛体2リンクによるパラメータに応じた投擲フォーム戦略の考察]{3次元剛体2リンクによるパラメータに応じた投擲フォーム戦略の考察}

\section{はじめに}
前章では2次元モデルでの投擲フォーム戦略の考察を行ったが、3次元に拡張することにより、より人間に近いモデルでの投擲フォーム戦略の考察が可能となると考えられる。そこで本章では、3次元モデルによる投擲物の重さに応じた投擲フォームの比較により、考察した投擲フォーム戦略について述べる。
\section{動力学モデル}
本章で用いた動力学モデルを\figref{5.1.jpg}に示す。本章では、\figref{5.2.jpg}で示す物理エンジンMuJoCoに標準搭載されているhumanoidモデル「Unitree G1」を改変し、人間の腕を肩関節3自由度と肘関節1自由度の計4自由度から構成される3次元剛体2リンクとしてモデル化した。なお、\figref{5.1.jpg}において、肩関節から肘関節までを上腕リンク、肘関節から手先までを前腕リンクとし、手首や指の自由度は0とした。また、体幹リンクも自由度0とし、世界座標に固定した。
\fig{5.1.jpg}{width=0.5\hsize}{動力学モデル}
\fig{5.2.jpg}{width=0.5\hsize}{Unitree G1}
\section{シミュレータの作成}
3次元剛体2リンクを強化学習するためのQ学習シミュレータをpythonで実装した。運動方程式は、MuJoCoで内部的に解いた。runge-kutta法により数値積分し運動学を解くことで、3次元剛体2リンクの角度や角速度を計算する。
\section{可動範囲}
\section{強化学習の設定}
\subsection{状態}
状態変数は8つとし、肩関節ピッチ軸まわりの角度$\theta_{p}$、角速度$\dot{\theta}_{p}$,肩関節ロール軸まわりの角度$\theta_{r}$、角速度$\dot{\theta}_{r}$,肩関節ヨー軸まわりの角度$\theta_{y}$、角速度$\dot{\theta}_{y}$,肘関節の角度$\theta_{e}$、角速度$\dot{\theta}_{e}$とした。\\
各関節の角度については、5.4章の通りである。
また、各関節の角速度については、-10.0 m/s $\le$ $\dot{\theta}_{1}$ $\le$ 10.0 m/s、-10.0 m/s $\le$ $\dot{\theta}_{2}$ $\le$ 10.0 m/sとした。
分割数は各角度が5分割、各角速度が2分割であり、全ての状態を$5^{4}\times 2^{4}$=10000通りで表すことができる。
\subsection{行動}
行動は、全625通りに設定した。肩関節にかかるトルクを正2通り、0、負2通りの計5通り、同様に肘関節にかかるトルクも正2通り、0、負2通りの計5通りとした。
これにより、Qテーブルは10000$\times$625=6250000通りで表すことができる。
\subsection{報酬}
本章では、投擲物の飛距離を報酬とした。投擲物のモデル化は行っていないため、投射中の投擲物に生じる空気抵抗等は考慮しないものとする。\\
飛距離の計算にあたり、ピッチ、ロール、ヨーの3方向成分の手先速度をMuJoCoより取得し、それぞれ$v_{p}$、$v_{r}$、$v_{y}$とした。手先速度成分の合成は、

\begin{eqnarray}
  \equlabel{5.1}
  v_{syn} = \sqrt{v_{p}^{2} + v_{r}^{2} + v_{y}^{2}}
\end{eqnarray}

\begin{eqnarray}
  \equlabel{5.2}
  v_{pr} = \sqrt{v_{p}^{2} + v_{r}^{2}}
\end{eqnarray}

であり、\equref{5.1}、\equref{5.2}より投射角$\theta_{v}$は、

\begin{eqnarray}
  \equlabel{5.3}
  \theta_{v} = \arctan2(\frac{v_{y}}{v_{pr}})
\end{eqnarray}

各ステップ時のヨー軸成分の座標$h_{y}$を手先高さとする。しかし、この座標は肩関節を原点とした時の値であり、本来の手先高さは地面から肩関節までの高さを考慮する必要がある。よって、身長を$L$としたときの手先高さ$h$は、

\begin{eqnarray}
  \equlabel{5.4}
  h = 0.818L + h_{y}
\end{eqnarray}

リリース時の手先高さを考慮した投射時間$t$は、

\begin{eqnarray}
  \equlabel{5.5}
  t = \frac{v_{syn}\sin\theta_{v} + \sqrt{{v_{syn}}^2\sin^2\theta_{syn} + 2gh}}{g}
\end{eqnarray}

\subsection{その他}
\section{投擲物の重さに応じた最適投擲フォームの比較}
\subsection{シミュレーション設定}
\section{結果・考察}
