
\chapter[3次元剛体2リンクによるパラメータに応じた投擲フォームの導出・比較と戦略の考察]{3次元剛体2リンクによるパラメータに応じた投擲フォームの導出・比較と戦略の考察}

\section{はじめに}
前章では2次元モデルでの投擲フォーム戦略の考察を行ったが,3次元に拡張することにより,より人間に近いモデルでの投擲フォーム戦略の考察が可能となると考えられる.そこで本章では,3次元モデルによる投擲物の重さに応じた投擲フォームの比較により,考察した投擲フォーム戦略について述べる.
\section{動力学モデル}
本章で用いた動力学モデルを\figref{5.1.eps}に示す.本章では,\figref{5.2.eps}で示す物理エンジンMuJoCo\cite{mujoco}\cite{mujoco2}に標準搭載されているhumanoidモデル「Unitree G1」\cite{unitreeg1}を改変し,人間の腕を肩関節3自由度と肘関節1自由度の計4自由度から構成される3次元剛体2リンクとしてモデル化した.なお,\figref{5.1.eps}において,肩関節から肘関節までを上腕リンク,肘関節から手先までを前腕リンクとし,手首や指の自由度は0とした.また,体幹リンクも自由度0とし,世界座標に固定した.

\fig{5.1.eps}{width=0.8\hsize}{3D rigid 2 links(stabilization of the trunk)}
\fig{5.2.eps}{width=0.8\hsize}{Unitree G1}

\section{シミュレータの作成}
3次元剛体2リンクを強化学習するためのQ学習シミュレータをpythonで実装した.運動方程式は,MuJoCoで内部的に解いた.Runge-Kutta法により数値積分し運動学を解くことで,3次元剛体2リンクの角度や角速度を計算する.
\section{可動範囲}
本章における可動範囲を\tabref{5.1}に示す.それぞれの可動範囲は,各自由度のそれぞれの静的な可動範囲\cite{range}に基づいて設定した.\figref{5.6.eps}が各自由度0 deg の際の各平面からの姿勢であり,MuJoCoにおける回転方向は右ねじの法則に基づいて定義されている.各軸の回転方向は次の通りである.

\begin{itemize}
  \item 肩関節ピッチ軸周りの回転:肩関節の伸展方向が正,屈曲方向が負
  \item 肩関節ロール軸周りの回転:肩関節の内転方向が正,外転方向が負
  \item 肩関節ヨー軸周りの回転:肩関節の内旋方向が正,外旋方向が負
  \item 肘関節周りの回転:肩関節の伸展方向が正,屈曲方向が負
\end{itemize}

\figt{5.6.eps}{width=1.0\hsize}{Model viewed from each plane}

\begin{table}[tb]
  \begin{center}
    \caption{Range of motion: shoulder 3 degrees of freedom, elbow 1 degree of freedom}
    \tablabel{5.1}
    \begin{tabular}{c|c|c|c}
      \hline
      Freedom & Unit & Min Range & Max Range \\
      \hline
      Shoulder pitch & deg & $-135$ & 45 \\
      Shoulder roll & deg & $-135$ & $-35$ \\
      Shoulder yaw & deg & $-150$ & 180 \\
      Elbow pitch & deg & $-20$ & 90 \\
      \hline
    \end{tabular}
  \end{center}
\end{table}

\tabref{5.1}の可動範囲は各自由度ごとに見ると人間が可能な姿勢である.しかし人間の各関節は互いに干渉するため,組み合わせ次第では人間が不可能な姿勢となる.よって,肩関節ピッチ軸周り,ロール軸周り,ヨー軸周り,肘関節周りの順に,可動範囲を人間が可能な姿勢に収まるように設定した.\\
 まず,肩関節ピッチ軸周りの可動範囲に基づいた肩関節ロール軸周りの可動範囲について確認したところ全ての可動範囲で人間が可能な姿勢であったため,そのまま可動範囲として採用した.次に肩関節ピッチ軸周りとロール軸周りに基づいた可動範囲について確認したところ,組み合わせ次第で人間が不可能な姿勢を取ることを確認した.そこで,スプライン補間\cite{spline}を用いて肩関節ピッチ軸周りとロール軸周りの角度によって可動範囲が変動するように設定した.最後に肩関節に基づいた肘関節について確認したところ全ての可動範囲で人間が可能な姿勢であったため,そのまま可動範囲として設定した.以上の設定により,全ての関節角度の組み合わせで人間が可能な姿勢を取ることができる.なお,計算精度上,数 deg 可動範囲外の角度を取ることがあるが,異常姿勢ではないため問題ないとした.


\section{強化学習の設定}
\subsection{状態}
状態変数は8つとし,肩関節ピッチ軸周りの角度$\theta_{p}$,角速度$\dot{\theta}_{p}$,肩関節ロール軸周りの角度$\theta_{r}$,角速度$\dot{\theta}_{r}$,肩関節ヨー軸周りの角度$\theta_{y}$,角速度$\dot{\theta}_{y}$,肘関節の角度$\theta_{e}$,角速度$\dot{\theta}_{e}$とした.\\
 各関節の角度については\tabref{5.1}の通りであり,肩関節ロール軸周りのみスプライン補間に基づく範囲とした.
また,各関節の角速度については,$-10.0$ m/s $\le$ $\dot{\theta}_{1}$ $\le$ 10.0 m/s,$-10.0$ m/s $\le$ $\dot{\theta}_{2}$ $\le$ 10.0 m/sとした.
分割数は各角度が5分割,各角速度が2分割であり,全ての状態を$5^{4}\times 2^{4}=10000$通りで表すことができる.
\subsection{行動}
行動は,全625通りに設定した.肩関節にかかるトルクを正2通り,0,負2通りの計5通り,同様に肘関節にかかるトルクも正2通り,0,負2通りの計5通りとした.
これにより,Qテーブルは$10000 \times 625=6250000$通りで表すことができる.
\subsection{報酬}
本章では,投擲物の飛距離を報酬とした.投擲物のモデル化は行っていないため,投射中の投擲物に生じる空気抵抗等は考慮しないものとする.\\
飛距離の計算にあたり,ピッチ,ロール,ヨーの3方向成分の手先速度をMuJoCoより取得し,それぞれ$v_{p}$,$v_{r}$,$v_{y}$とした.手先速度成分の合成は,

\begin{eqnarray}
  \equlabel{5.1}
  v_{syn} = \sqrt{{v_{p}}^{2} + {v_{r}}^{2} + {v_{y}}^{2}}
\end{eqnarray}

\begin{eqnarray}
  \equlabel{5.2}
  v_{pr} = \sqrt{{v_{p}}^{2} + {v_{r}}^{2}}
\end{eqnarray}

であり,\equref{5.1},\equref{5.2}より投射角$\theta_{v}$は,

\begin{eqnarray}
  \equlabel{5.3}
  \theta_{v} = \arctan2(\frac{v_{y}}{v_{pr}})
\end{eqnarray}

各ステップ時のヨー軸成分の座標$h_{y}$を手先高さとする.しかし,この座標は肩関節を原点とした時の値であり,本来の手先高さは地面から肩関節までの高さを考慮する必要がある.よって,身長を$L$としたときの手先高さ$h$は,

\begin{eqnarray}
  \equlabel{5.4}
  h = 0.818L + h_{y}
\end{eqnarray}

リリース時の手先高さを考慮した投射時間$t$は,

\begin{eqnarray}
  \equlabel{5.5}
  t = \frac{v_{syn}\sin\theta_{v} + \sqrt{{v_{syn}}^2\sin^2\theta_{syn} + 2gh}}{g}
\end{eqnarray}


以上より,飛距離$D$は\equref{5.1},\equref{5.3},\equref{5.5}を用いて,

\begin{eqnarray}
  \equlabel{5.6}
  D = v_{syn} \cdot \cos\theta_{v} \cdot t
\end{eqnarray}

この報酬により,より直接的に遠投を行うための投擲フォームを導出することができる.また,2次元のときと同様に,罰則として累積消費エネルギーを採用し,\equref{5.11}で計算した.

\begin{eqnarray}
  \equlabel{5.7}
  CE_{p}&=&\int \tau_{p}(t)\cdot\dot{\theta}_{p} dt\\
        &\approx&|\tau_{p}(t)\cdot\dot{\theta}_{p}|\cdot\Delta t
\end{eqnarray}

\begin{eqnarray}
  \equlabel{5.8}
  CE_{r}&=&\int \tau_{r}(t)\cdot\dot{\theta}_{r} dt\\
        &\approx&|\tau_{r}(t)\cdot\dot{\theta}_{r}|\cdot\Delta t
\end{eqnarray}

\begin{eqnarray}
  \equlabel{5.9}
  CE_{y}&=&\int \tau_{y}(t)\cdot\dot{\theta}_{y} dt\\
        &\approx&|\tau_{y}(t)\cdot\dot{\theta}_{y}|\cdot\Delta t
\end{eqnarray}

$\tau_{p},\tau_{r},\tau_{y}$はそれぞれ肩関節ピッチ軸周り,ロール軸周り,ヨー軸周りに与えるトルクである.また,$CE_{p},CE_{r},CE_{y}$はそれぞれ肩関節ピッチ軸周り,ロール軸周り,ヨー軸周りの累積消費エネルギーであり,肩関節の累積消費エネルギーの合計は,

\begin{eqnarray}
  \equlabel{5.10}
  CE_{s} = CE_{p} + CE_{r} + CE_{y}
\end{eqnarray}

肘関節の累積消費エネルギーは,

\begin{eqnarray}
  \equlabel{5.11}
  CE_{e}&=&\int \tau_{e}(t)\cdot\dot{\theta}_{e} dt\\
        &\approx&|\tau_{e}(t)\cdot\dot{\theta}_{e}|\cdot\Delta t
\end{eqnarray}

$\tau_{e}$は肘関節に与えるトルクである.よって,全体の累積消費エネルギーは,

\begin{eqnarray}
  \equlabel{5.12}
  CE = CE_{s} + CE_{e}
\end{eqnarray}

以上より,報酬の設計は,
\begin{eqnarray}
  \equlabel{5.13}
  Reward = D - scale \times CE
\end{eqnarray}

\subsection{その他}
本章では,\equref{Q}における学習率$\alpha$を0.1,割引率$\gamma$を0.9に設定した.これは将来の報酬を優先的に考慮した学習を期待するためである.
1タイムステップは0.001 sであり,1エピソード内のステップ数は3000,すなわち,3 sとした.エピソード数に関しては,10000エピソードの中で最も報酬が高いエピソードを採用した.

\section{投擲物の重さに応じた最適投擲フォームの比較}
\subsection{シミュレーション設定}
本検証で用いたパラメータを\tabref{5.2}に示す.身体のサイズは,身長1.72 m,体重70 kgの人間の各部位のサイズとした.
投擲物は,野球の硬式球と砲丸の重さを参考に,0.14 kg と7.24 kg とした.
\begin{table}[tb]
  \begin{center}
    \caption{Parameters which are used for 3D rigid 2 links simulation(0.14 kg vs 7.24 kg)}
    \tablabel{5.2}
    \begin{tabular}{c|c|c}
      \hline
      Parameters & Unit & Values \\
      \hline
      $l_{1}$ & m & 0.32 \\
      $l_{2}$ & m & 0.44 \\
      $l_{f}$ & m & 0.25 \\
      $l_{h}$ & m & 0.19 \\
      $m_{f}$ & kg & 1.12 \\
      $m_{h}$ & kg & 0.42 \\
      $m_{o}$ & kg & 0.14 or 7.24 \\
      $m_{1}$ & kg & 1.96 \\
      $m_{2}$ & kg & 1.68 or 8.78 \\
      \hline
    \end{tabular}
  \end{center}
\end{table}
重力加速度$g$を9.8 $\mathrm{m/s}^{2}$とする.また,前章と同様に,前腕と手と投擲物の要素を合わせて前腕リンクとしてみなすため,3つの要素を考慮して設定を行った.
前腕の要素として,$l_{f}$を前腕長さ,$m_{f}$を前腕重さ,$l_{h}$を手長さ,$m_{h}$を手重さとした.\\
 次に,野球の硬式球の重さの投擲物と砲丸重さの投擲物のときのそれぞれの上腕・前腕リンクの重心までの長さ,上腕・前腕リンクの慣性モーメント,肩関節の3軸周り・肘関節に与えるトルク,肩関節の3軸周り・肘関節の粘性係数について,表\tabref{5.3}に示す.各リンクの重心までの長さと慣性モーメントについての計算は,\equref{2.26}から\equref{2.31}を基に行った.各関節に与えるトルクについて,肩関節ピッチ軸周りとロール軸周り,肘関節はそれぞれ5種類のトルクから選択される.一方,肩関節ヨー軸周りのトルクについては,スプライン補間によって設定した可動範囲に基づき,可動範囲内にトルクが収まるように与えるトルクも補完する設定とした.

\begin{table}[tb]
  \begin{center}
    \caption{Calculation Results of $l_{g}$,$I$,$\tau$,$b$ for 3D rigid 2 links(0.14 kg vs 7.24 kg)}
    \tablabel{5.3}
    \begin{tabular}{c|c|c|c}
      \hline
      Parameters & unit & Values of o.14 kg & Values of 7.24 kg \\
      \hline
      $l_{g1}$ & m & 0.16 & 0.16 \\
      $l_{g2}$ & m & 0.21 & 0.40 \\
      $I_{1}$ & kg $\cdot$ $\mathrm{m}^2$ & 0.017 & 0.017 \\
      $I_{2}$ & kg $\cdot$ $\mathrm{m}^2$ & 0.15 & 1.59 \\
      $\tau_{p}$ & N$\cdot$m & $-40, -20, 0, +20, +40$ & $-40, -20, 0, +20, +40$ \\
      $\tau_{y}$ & N$\cdot$m & $-40, -20, 0, +20, +40$ & $-40, -20, 0, +20, +40$ \\
      $\tau_{e}$ & N$\cdot$m & $-30, -15, 0, +15, +30$ & $-30, -15, 0, +15, +30$ \\
      $b_{p}$ &  & 1.0 & 1.0\\
      $b_{r}$ &  & 0.8 & 0.8\\
      $b_{y}$ &  & 0.5 & 0.5\\
      $b_{e}$ &  & 0.2 & 0.2\\
      \hline
    \end{tabular}
  \end{center}
\end{table}

以上の設定をふまえ,投擲物に応じた投擲フォームの強化学習を行った際の報酬の遷移を\figref{5.9.eps}に示す.重さ0.14 kg の投擲物を遠投する投擲フォームは2983エピソード,重さ7.24 kg の投擲物を遠投する投擲フォームは9140エピソードを採用した.

\fig{5.9.eps}{width=1.0\hsize}{Reward progress for 3D rigid 2 links(left: 0.14 kg thrown object,right: 7.24 kg \newline     thrown object)}

\subsection{結果・考察}
初期状態について,$\theta_{p}$,$\theta_{r}$,$\theta_{y}$,$\theta_{e}$はランダム値,$\dot{\theta}_{p}$,$\dot{\theta}_{r}$,$\dot{\theta}_{y}$,$\dot{\theta}_{e}$は0とした.これにより,最も報酬が高くなる時の初期姿勢を導出することができる.投擲物に応じた遠投するための投擲フォーム戦略の結果・考察を以下に示す.
\figref{5.11.eps},\figref{5.12.eps}は強化学習により最適化した,重さ0.14 kg の投擲物を遠投するための投擲フォームである.\figref{5.11.eps}はピッチ軸に垂直な面から見た投擲フォーム,\figref{5.12.eps}はロール軸に垂直な面から見た投擲フォームである.また,\figref{5.13.eps},\figref{5.14.eps}は強化学習により最適化した,重さ7.24 kg の投擲物を遠投するための投擲フォームである.\figref{5.13.eps}はピッチ軸に垂直な面から見た投擲フォーム,\figref{5.14.eps}はロール軸に垂直な面から見た投擲フォームである.リリース瞬間は,1エピソードの中で最も報酬が高いステップとした.投擲フォームは投擲開始からリリースまでとし,リリース瞬間の肩関節3軸周りと肘関節の角度,手先の高さ,手先速度,投射角,投擲物の飛距離を\tabref{5.4}に示す.重さ0.14 kg の投擲物の投擲フォームのリリース瞬間は0.489 s,重さ7.24 kg の投擲物の投擲フォームのリリース瞬間は0.447 sであった.

\begin{table}[tb]
  \begin{center}
    \caption{Data at the moment of release for 3D rigid 2 links(0.14 kg vs 7.24 kg)}
    \tablabel{5.4}
    \begin{tabular}{c|c|c|c}
      \hline
      Parameters & unit & Values of 0.14 kg & Values of 7.24 kg \\
      \hline
      $\theta_{p}$ & deg & $-125.11$ & $-97.58$ \\
      $\theta_{r}$ & deg & $-136.91$ & $-46.48$ \\
      $\theta_{y}$ & deg  & 36.30 & 0.00 \\
      $\theta_{e}$ & deg & $-23.52$ & 20.25 \\
      $h$ & m & 1.96 & 2.03 \\
      $v_{syn}$ & m/s & 26.00 & 6.80 \\
      $\theta_{v}$ & deg & 48.27 & 36.91 \\
      $D$ & m & 70.17 & 6.42 \\
      \hline
    \end{tabular}
  \end{center}
\end{table}

\figref{5.7.eps}は投擲開始からリリースまでの肩関節3軸周り,肘関節周りのトルクの推移,\figref{5.21.eps}は投擲開始からリリースまでの手先速度の推移である.\figref{5.7.eps},\figref{5.21.eps}において,左図は重さ0.14 kg の投擲物を遠投するための投擲フォームの各時系列,右図は重さ7.24 kg の投擲物を遠投するための投擲フォームの各時系列である.\\
 まず,重さ0.14 kg の投擲物を遠投するための投擲フォームについて述べる.初期姿勢は肘を伸ばして横に腕を広げた状態であったが,そこから主に肩関節ピッチ軸周りのトルクによって腕が後方に引かれている.その際,0.2 s あたりから,肘関節のトルクは肘が屈曲する方向へ入力される.その後,腕を引き切った0.35 s あたりから肩関節ヨー軸周りのトルクによって腕を前方と振り上げてリリースを迎える.その際肩関節ピッチ軸周りのトルクに注目すると正負の値を繰り返しているが,これは運動連鎖の活用の影響と考えられる.腕を前方に回転する際,肩関節ピッチ軸周りのトルクを腕の回転方向に入力し続けることで,より腕の回転に貢献できるように見える.しかし,重力の影響で最も腕を回転させるためにトルクが必要な0.35 s あたりから腕の回転とは逆方向にトルクが入力されている.これは肩関節ピッチ軸周りの関節が持つエネルギーをヨー軸周りの関節にエネルギーを伝達していると考えられる.\\
 次に,重さ7.24 kg の投擲物を遠投するための投擲フォームについて述べる.初期姿勢は肘関節を屈曲させて腕を前方高く上げている.これは,慣性モーメントの影響が考えられる.重い投擲物を持って腕を後方へ回転させると慣性モーメントの影響で肩関節や肘関節にかかる負荷が大きい.そのため,あらかじめ手先位置の高い姿勢を取ることで腕を振り上げる動作を回避しようとしていると考えられる.投擲開始から腕を後方に引く動作は無く,主に肩関節ロール軸周りのトルクによって腕が前方へ動く.この際,肘関節は肘を屈曲させる方向へトルクが入力され続けており,慣性モーメントによる腕への負担を軽減しようとしていると考えられる.0.256 s あたりから肩関節ロール軸周りと肘関節のトルクの入力が負から正となり,ロール軸周りのトルクが正から負となる.これは,重さ0.14 kg の投擲物の際と同様に,運動連鎖による前腕へのエネルギーの伝達により,肘を伸展させるエネルギーを大きくしていると考えられる.その後リリースを迎えるが,投擲方向は正面ではなく左方向となった.これは,本検証で用いた動力学モデル手首の自由度が0であること,ロール軸方向に正であれば投擲方向は問わない報酬設定の影響があると考えられる.\\
 以上より,重さ0.14 kg の投擲物の投擲フォームと重さ7.24 kg の投擲物の投擲フォームの比較よりみられる戦略の違いは,「腕の押し出し度合い」であると考えられる.\figref{5.15.eps}は投擲開始からリリースまでの,肩を基準とした際の手先高さの推移である.\figref{5.15.eps}において,左図は重さ0.14 kg の投擲物を遠投するための投擲フォームの各時系列,右図は重さ7.24 kg の投擲物を遠投するための投擲フォームの各時系列である.重さ0.14 kg の投擲物の際はリリース前に一度手先が下がってから一気に高くなるが,重さ7.24 kg の投擲物の際はほば手先高さに変化がみられない.この波形の直線度が高いほど腕の押し出し度合いが高い.よって,重さ0.14 kg の投擲物の投擲フォームは,運動連鎖による肩関節の回転を重要視した戦略が考えられる.一方,重さ7.24 kg の投擲物の投擲フォームは,肘の伸展や前腕を重要視した腕の押し出し度合いの高い戦略が考えられる.

\fig{5.11.eps}{width=1.0\hsize}{Throwing form from start to release for 3D rigid 2 links from a plane perpendicular \newline     to the pitch axis(0.14 kg thrown object,1.72 m tall human)}
\fig{5.12.eps}{width=1.0\hsize}{Throwing form from start to release for 3D rigid 2 links from a plane perpendicular \newline     to the roll axis(0.14 kg thrown object)}
\fig{5.13.eps}{width=1.0\hsize}{Throwing form from start to release for 3D rigid 2 links from a plane perpendicular \newline     to the pitch axis(7.24 kg thrown object)}
\fig{5.14.eps}{width=1.0\hsize}{Throwing form from start to release for 3D rigid 2 links from a plane perpendicular \newline     to the roll axis(7.24 kg thrown object)}
\fig{5.7.eps}{width=1.0\hsize}{Transition of shoulder 3 degrees of freedom and elbow 1 degree of freedom torque \newline     from start to release for 3D rigid 2 links(left: 0.14 kg thrown object,right: 7.24 \newline     kg thrown object)}
% \fig{5.20.eps}{width=1.0\hsize}{Transition of shoulder 3 degrees of freedom and elbow 1 degree of freedom angular velocity from start to release(left: 0.14 kg thrown object,right: 7.24 kg thrown object)}
\fig{5.21.eps}{width=1.0\hsize}{Transition of hand speed from start to release for 3D rigid 2 links\newline     (left: 0.14 kg thrown object,right: 7.24 kg thrown object)}
\fig{5.15.eps}{width=1.0\hsize}{Transition of hand height relative shoulder from start to release for 3D rigid \newline     2 links(left: 0.14 kg thrown object,right: 7.24 kg thrown object)}

\clearpage
\section{腕の長さに応じた最適投擲フォームの比較}
\subsection{シミュレーション設定}
本検証で用いたパラメータを\tabref{5.5}に示す.身体のサイズは,身長1.72 m,体重70 kgの人間と,身長1.90 m,体重70 kg の各部位のサイズとした.
投擲物は,野球の硬式球の重さを参考に,0.14 kg とした.

\begin{table}[tb]
  \begin{center}
    \caption{Parameters which are used for 2link-4DOF simulation}
    \tablabel{5.5}
    \begin{tabular}{c|c|c}
      \hline
      Parameters & Unit & Values \\
      \hline
      $l_{1}$ & m & 0.32 or 0.35 \\
      $l_{2}$ & m & 0.44 or 0.48 \\
      $l_{f}$ & m & 0.25 or 0.28 \\
      $l_{h}$ & m & 0.19 or 0.20 \\
      $m_{f}$ & kg & 1.12 \\
      $m_{h}$ & kg & 0.42 \\
      $m_{o}$ & kg & 0.14 \\
      $m_{1}$ & kg & 1.96 \\
      $m_{2}$ & kg & 1.68 \\
      \hline
    \end{tabular}
  \end{center}
\end{table}

重力加速度$g$を9.8 $\mathrm{m/s}^{2}$とする.また,前章と同様に,前腕と手と投擲物の要素を合わせて前腕リンクとしてみなすため,3つの要素を考慮して設定を行った.
前腕の要素として,$l_{f}$を前腕長さ,$m_{f}$を前腕重さ,$l_{h}$を手長さ,$m_{h}$を手重さとした.\\

次に,野球の硬式球の重さの投擲物と砲丸重さの投擲物のときのそれぞれの上腕・前腕リンクの重心までの長さ,上腕・前腕リンクの慣性モーメント,肩関節の3軸周り・肘関節に与えるトルク,肩関節の3軸周り・肘関節の粘性について,表\tabref{5.6}に示す.各リンクの重心までの長さと慣性モーメントについての計算は,\equref{2.26}から\equref{2.31}を基に行った.各関節に与えるトルクについて,肩関節ピッチ軸周りとロール軸周り,肘関節はそれぞれ5種類のトルクから選択される.一方,肩関節ヨー軸周りのトルクについては,スプライン補間によって設定した可動範囲に基づき,可動範囲内にトルクが収まるように与えるトルクも補完する設定とした.

\begin{table}[tb]
  \begin{center}
    \caption{Calculation Results of $l_{g}$,$I$,$\tau$,$b$}
    \tablabel{5.6}
    \begin{tabular}{c|c|c|c}
      \hline
      Parameters & unit & Values of baseball & Values of cannonball \\
      \hline
      $l_{g1}$ & m & 0.16 & 0.18 \\
      $l_{g2}$ & m & 0.21 & 0.23 \\
      $I_{1}$ & kg $\cdot$ $\mathrm{m}^2$ & 0.017 & 0.018 \\
      $I_{2}$ & kg $\cdot$ $\mathrm{m}^2$ & 0.020 & 0.099 \\
      $\tau_{p}$ & N$\cdot$m & $-40, -20, 0, 20, 40$ & $-40, -20, 0, 20, 40$ \\
      $\tau_{r}$ & N$\cdot$m & $-40, -20, 0, 20, 40$ & $-40, -20, 0, 20, 40$ \\
      $\tau_{y}$ & N$\cdot$m & $-40, -20, 0, 20, 40$ & $-40, -20, 0, 20, 40$ \\
      $\tau_{e}$ & N$\cdot$m & $-30, -15, 0, 15, 30$ & $-30, -15, 0, 15, 30$ \\
      $b_{p}$ &  & 1.0 & 1.0\\
      $b_{r}$ &  & 0.8 & 0.8\\
      $b_{y}$ &  & 0.5 & 0.5\\
      $b_{e}$ &  & 0.2 & 0.2\\
      \hline
    \end{tabular}
  \end{center}
\end{table}

以上の設定をふまえ,投擲物に応じた投擲フォームの強化学習を行った際の報酬の遷移を\figref{5.10.eps}に示す.身長1.72 m の人間に基づいた腕の長さによる遠投をするための投擲フォームは2983エピソード,身長1.90 m の人間に基づいた腕の長さによる遠投をするための投擲フォームは3060エピソードを採用した.

\figt{5.10.eps}{width=1.0\hsize}{Reward progress(left:1.72 m tall human,right:1.90 m tall human)}

\subsection{結果・考察}
初期状態について,投擲物の重さによる投擲フォームの比較と同様に$\theta_{p}$,$\theta_{r}$,$\theta_{y}$,$\theta_{e}$はランダム値,$\dot{\theta}_{p}$,$\dot{\theta}_{r}$,$\dot{\theta}_{y}$,$\dot{\theta}_{e}$は0とした.腕の長さに応じた遠投をするための投擲フォーム戦略の結果・考察を以下に示す.
\figref{5.11.eps},\figref{5.12.eps}は強化学習により最適化した,身長1.72 m の人間に基づいた腕の長さによる遠投をするための投擲フォームである.\figref{5.11.eps}はピッチ軸に垂直な面から見た投擲フォーム,\figref{5.12.eps}はロール軸に垂直な面から見た投擲フォームである.また,\figref{5.16.eps},\figref{5.17.eps}は強化学習により最適化した,身長1.90 m の人間に基づいた腕の長さによる遠投をするための投擲フォームである.\figref{5.16.eps}はピッチ軸に垂直な面から見た投擲フォーム,\figref{5.17.eps}はロール軸に垂直な面から見た投擲フォームである.リリース瞬間は,1エピソードの中で最も報酬が高いステップとした.投擲フォームは投擲開始からリリースまでとし,リリース瞬間の肩関節3軸周りと肘関節の角度,手先の高さ,手先速度,投射角,投擲物の飛距離を\tabref{5.7}に示す.身長1.72 m の人間に基づいた腕の長さによる遠投をするための投擲フォームのリリース瞬間は0.489 s,身長1.90 m の人間に基づいた腕の長さによる遠投をするための投擲フォームのリリース瞬間は1.109 sであった.

\begin{table}[tb]
  \begin{center}
    \caption{Data at the moment of release}
    \tablabel{5.7}
    \begin{tabular}{c|c|c|c}
      \hline
      Parameters & unit & Values of 1.72 m & Values of 1.90 m \\
      \hline
      $\theta_{p}$ & deg & $-125.11$ & $-102.55$ \\
      $\theta_{r}$ & deg & $-136.91$ & $-135.31$ \\
      $\theta_{y}$ & deg  & 36.30 & 65.46 \\
      $\theta_{e}$ & deg & $-23.52$ & $-8.10$ \\
      $h$ & m & 1.96 & 2.04 \\
      $v_{syn}$ & m/s & 26.00 & 29.87 \\
      $\theta_{v}$ & deg & 48.27 & 47.92 \\
      $D$ & m & 70.17 & 92.16 \\
      \hline
    \end{tabular}
  \end{center}
\end{table}

\figref{5.8.eps}は投擲開始からリリースまでの肩関節3軸周り,肘関節周りのトルクの推移,\figref{5.8.eps}は投擲開始からリリースまでの肩関節3軸周り,肘関節周りの角速度の推移,\figref{5.8.eps}は投擲開始からリリースまでの手先速度の推移である.\figref{5.8.eps}において,左図は身長1.72 m の人間に基づいた腕の長さによる遠投をするための投擲フォームの各時系列,右図は身長1.90 m の人間に基づいた腕の長さによる遠投をするための投擲フォームの各時系列である.\\
比較の結果,身長1.72 m の人間に基づいた腕の長さの 0.28 s からリリースまでの投擲フォームと身長1.90 m の人間に基づいた腕の長さの 0.80 s からリリースまでの投擲フォームが類似していた.また,\figref{5.25.eps}は\figref{5.8.eps}の中でも投擲フォームが類似していた区間に着目したトルクの時系列であるが,各関節類似した傾向がみられた.よって,腕の長さに問わず遠投をするための投擲フォームの戦略は同様であると考えられる.しかし,身長1.90 m の人間に基づいた腕の長さによる遠投をするための投擲フォームは 0.80 s の姿勢になるまでに,身長1.72 m の人間に基づいた腕の長さによる遠投をするための投擲フォームとは異なる挙動がみられた.これは,慣性モーメントの影響が考えられる.腕が長くなるとモーメントアームが大きくなるため,腕を回転させるのに必要なトルクが大きくなる.本検証で用いたトルクは身長1.72 m の人間に基づいた腕を回転させるのに十分なトルクであり,身長1.90 m の人間に基づいた腕をそのまま後方に引くにはトルクが不十分であったと考えられる.肘関節に入力されるトルクが約 0.3 s まで負の値,つまり肘の角速度が負となる方向であり,そこからトルクが正の値となり肘の角速度が正となる.これは肘を屈曲させてから伸展させる動作であり,この動作によって腕を振り上げるためのエネルギーを大きくしていると考えられる.\\
 以上より,身長1.72 m の人間に基づいた腕の長さによる遠投をするための投擲フォームと身長1.90 m の人間に基づいた腕の長さによる遠投をするための投擲フォームの比較によりみられる戦略は,ともに運動連鎖による肩関節の回転を重要視した戦略が考えられる.

\clearpage
\figt{5.16.eps}{width=1.0\hsize}{Throwing form from start to release from a plane perpendicular to the pitch axis(1.90 m tall human)}
\fig{5.17.eps}{width=1.0\hsize}{Throwing form from start to release from a plane perpendicular to the roll axis(1.90 m tall human)}
\fig{5.8.eps}{width=1.0\hsize}{Transition of shoulder and elbow torque from start to release(left:1.72 m tall human,right:1.90 m tall human)}
\fig{5.23.eps}{width=1.0\hsize}{Transition of shoulder 3 degrees of freedom and elbow 1 degree of freedom angular velocity from start to release(left:1.72 m tall human,right:1.90 m tall human)}
\fig{5.24.eps}{width=1.0\hsize}{Transition of hand speed from start to release(left:1.72 m tall human,right:1.90 m tall human)}
\fig{5.25.eps}{width=1.0\hsize}{Transition of shoulder and elbow torque(left:1.72 m tall human from 0.28 s to release,right:1.90 m tall human from 0.80 s to release)}
