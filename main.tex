\documentclass[11pt]{jsk-thesis}

%レイアウト補正
%\usepackage{layout}
%\setlength \voffset {-1.0cm}
%\setlength \hoffset {-1.2cm}
%\setlength \textwidth {17cm}
%\setlength \textheight {24cm}

%パッケージ読み込み
\usepackage[dvipdfmx]{graphicx}
\usepackage[dvipdfmx]{color}
\usepackage{booktabs}
\usepackage{amsmath}
\usepackage{amssymb}
\usepackage{bm}
\usepackage{url}

%コマンド定義
%表の日付のフォントサイズ変更
\newcommand{\tdate}[1]{\scriptsize{#1}}
%単位"°"
%%\newcommand{\degree}[1]{#1^{\circ}}
%微分演算子関係
\newcommand{\dd}{\mathrm{d}}
\newcommand{\diff}[2]{\frac{\mathrm{d}#1}{\mathrm{d}#2}} %常微分
\newcommand{\diffline}[2]{\mathrm{d}#1/\mathrm{d}#2} %文章中での常微分
\newcommand{\ddiff}[3]{\frac{\mathrm{d}^#1 #2}{\mathrm{d} #3^#1}} %高階常微分
\newcommand{\ddiffline}[3]{\mathrm{d}^#1 #2/\mathrm{d} #2^#1} %文章中での高階常微分
\newcommand{\pdiff}[2]{\frac{\partial #1}{\partial #2}} %偏微分
\newcommand{\pddiff}[3]{\frac{\partial^#1 #2}{\partial #3^#1}} %高階偏微分
\newcommand{\non}[1]{#1^{*}} %無次元化変数

%関数
\newcommand{\erf}{\mathrm{erf}}

%その他
\newcommand{\myfig}[5]{
  \begin{figure}[#1]%
    \begin{center}%
      \includegraphics[width=#2]{#3}%
      \caption{#4}%
      \label{fig:#5}%
    \end{center}%
  \end{figure}%
}
%% \newcommand{\figref}[1]{Fig.\ref{fig:#1}}
%% \newcommand{\seclabel}[1]{\label{sec:#1}}
%% \newcommand{\secref}[1]{第{\bf\ref{sec:#1}}節}

\newcommand{\unit}[1]{\,[\mathrm{#1}]}

%見出し変更
\renewcommand{\figurename}{Fig.}
\renewcommand{\tablename}{Table}



\usepackage{ikuo}%%便利コマンド集.

\usepackage[dvipdfmx]{hyperref}  % 目次や参考文献をリンクにする。
\usepackage{pxjahyper} %% これを入れるとしおりが文字化けしない。out2uniが不要になる。
%% \hypersetup{bookmarksnumbered=true}
\hypersetup{colorlinks=true}
\hypersetup{linkcolor=black}
%% \hypersetup{linkbordercolor=black}
\hypersetup{urlcolor=black}
%% \hypersetup{urlbordercolor=black}
\hypersetup{citecolor=black}
%% \hypersetup{citebordercolor=black}

\usepackage{url} % \url のために必要。パッケージが無い人は探して入れる。
%% \url{http://nile.ulis.ac.jp/~yuka/}のようにして使う。

\newcommand{\FIGDIR}{./fig}        %図を置くディレクトリを指定する


\date{令和6年度卒業論文}
\title{卒論執筆に関する研究}
\author{指導教員 水内 郁夫 准教授 \\
\ \\
東京農工大学 \\
工学部 機械システム工学科 \\
\ \\
平成32年度入学\\
12345678\\
{\bf 山田 太郎}}

\begin{document}
\setlength{\baselineskip}{20pt}
\maketitle
\tableofcontents

%%各章は別ファイルにして以下にinculudeすると良い.
\chapter[序論]%
        {序論}
        \section{研究の背景と目的}
        投擲動作を行うスポーツは数多く存在するが,野球と砲丸投げのように競技によって投擲フォームは異なり,さらに同一競技内であっても個人によって投擲フォームは異なる.競技や個人によって投擲フォームが異なる要因として,投擲物や身体といった投擲フォームに関連するパラメータの違いが挙げられる.
        具体的なパラメータとして,投擲物は重さや大きさ,身体は慣性や各部位のサイズ等がある.これまで投擲フォームに関する研究例として,眞田の野球におけるオーバーハンドスローとサイドハンドスローの球速の比較\cite{sanada},大室らの野球における足の踏み出し幅による投球速度の比較\cite{omuro},黒松らの砲丸投げグライド投法における投擲フォーム改善前後の飛距離の比較\cite{kuromatsu}などがある.
        投擲フォームは球速や投擲物の飛距離等の,スポーツにおける総合性能に大きな影響を及ぼす.また,投擲に関する総合性能の研究例として,蔭山らの野球における体格や背筋力と投球速度の関係\cite{kageyama},\UTF{9AD9}橋らの野球における肩関節と股関節の可動域・筋力と投球速度の関係\cite{takahashi},坪井らの砲丸投げにおける投射速度・投射角と飛距離の関係\cite{tsuboi}などがある.スポーツにおいて総合性能向上は最も重要な要素の一つである.これらの研究はある一つの競技に特定した研究である.しかし,さまざまなパラメータに応じた投擲フォームがどのような戦略の基で成立しているのかに関する汎用的な理論は確立されていない.そこで,本研究ではシミュレーションにおいてさまざまなパラメータに応じた投擲フォームを導出・比較することで,さまざまな投擲フォームの戦略を検討・考察・議論することを目的とする.\\

        \section{本論文の構成}
        本論文では,全6章から構成される.以下に,各章の概要について述べる.
        \begin{itemize}
          \item 第1章(本章)では,研究の背景と目的について述べた.
          \item 第2章「強化学習を用いた投擲フォーム導出」では,本研究の学習手法として用いた強化学習について述べる.
          \item 第3章「剛体1リンクモデルによるリンク速度最適化シミュレーション」では,投擲フォームでの強化学習に先立ち,自作したシミュレータと手法の有用性について述べる.
          \item 第4章「2次元剛体2リンクによるパラメータに応じた投擲フォームの導出・比較と戦略の考察」では,腕に見立てた2次元剛体2リンクモデルを用いて投擲物の重さに応じた投擲フォーム戦略,腕の長さに応じた投擲フォーム戦略の結果・考察について述べる.
          \item 第5章「3次元剛体2リンクによるパラメータに応じた投擲フォームの導出・比較と戦略の考察」では,腕のモデルを3次元に拡張し,投擲物の重さに応じた投擲フォーム戦略の結果・考察,腕の長さに応じた投擲フォーム戦略の結果・考察について述べる.
          \item 第6章「結論および今後の展望」では,本研究で得られた結論および今後の展望について述べる.
        \end{itemize}

\include{fig_tab_equ}

\addcontentsline{toc}{chapter}{謝辞}
\markboth{謝辞}{謝辞}
\chapter*{謝辞}
本論文は,筆者が東京農工大学大学院生物システム応用科学府生物機能システム科学専攻博士前期課程に在学中に行った研究をまとめたものです.\\
 本論文をまとめるにあたり,丁寧なご指導ご鞭撻を賜った東京農工大学生物システム応用科学府生物機能システム科学専攻 水内郁夫教授に深く感謝の意を表します.水内先生には貴重なご意見や適切なご指導をいただき,充実した研究生活を送ることができました.また,目上の方に対する礼儀はもちろん,研究報告会等私の今後の人生に重要な知識・経験をご教授していただく機会も多く,大変貴重な日々でした.改めて心より感謝申し上げます.\\
 森下克幸助教にも深く感謝の意を表します.森下先生には普段から積極的にコミュニケーションを取っていただき,その都度貴重なご意見をいただきました.お気遣いいただいたこともあり,研究をより充実したものにしていただきました.\\
 本学在籍時に関わった研究室のメンバーには大変お世話になりました.特に同期の長田京右さん,菅野公景さん,高橋龍乃介さん,山本雄大さんは共に過ごす時間も長く大変お世話になりました.長田くんとは研究活動外でも共に行動することが多く,他愛もない会話をしていつも楽しませてくれました.菅野くんは頼りになる研究時と研究外のギャップが大きく,いつも笑わせてくれました.高橋くんとは話すたびにふざけ合って,疲れを吹き飛ばしてくれました.山本君は研究テーマも近く,また夜中に研究室で話をしたりして切磋琢磨することができました.また,先輩の赤羽聖さんにも大変お世話になりました.研究時の研究外でも頼れる兄貴分的な存在であり,いつも支えていただきました.また,M1とB4の後輩たちにも大変お世話になりました.先輩後輩の垣根を越えて交流をしてくれたおかげで,研究生活が一層楽しいものになりました.これからもより一層頑張ってください.\\
 最後に,学費の援助,生活の支援をしてくれた両親に,心より感謝申し上げます.学生生活で多くの経験をすることができたのも両親による支えがあったからであると感じています.本当にありがとうございました.


\addcontentsline{toc}{chapter}{参考文献}
\markboth{参考文献}{参考文献}
\bibliographystyle{junsrt}
\bibliography{reference}

\end{document}
