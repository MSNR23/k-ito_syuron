\chapter[序論]%
        {序論}
        \section{研究の背景と目的}
        \seclabel{intro}

        投擲動作を行うスポーツは数多く存在するが、野球と砲丸投げのように競技によって投擲フォームは異なり、さらに同一競技内であっても個人によって投擲フォームは異なる。競技や個人によって投擲フォームが異なる要因として、投擲物や身体といった投擲フォームに関連するパラメータの違いが挙げられる。
        具体的なパラメータとして、投擲物は重さや大きさ、身体は慣性や各部位のサイズ等がある。これまで投擲フォームに関する研究例として、眞田の野球におけるオーバーハンドスローとサイドハンドスローの球速の比較、大室らの野球における足の踏み出し幅による投球速度の比較、黒松らの砲丸投げグライド投法における投擲フォーム改善前後の飛距離の比較などがある。
        投擲フォームは球速や投擲物の飛距離等の、スポーツにおける総合性能に大きな影響を及ぼす。また、投擲に関する総合性能の研究例として、蔭山らの野球における年齢による体格や背筋力と投球速度の関係、高橋らの野球における肩関節と股関節の可動域・筋力と投球速度の関係、坪井らの砲丸投げにおける
        投射速度・投射角と飛距離の関係などがある。スポーツにおいて総合性能向上は最も重要な要素の一つである。これらの研究はある一つの競技に特定した研究である。しかし、さまざまなスポーツに応じた投擲フォームがどのような戦略の基で成立しているのかに関する汎用的な理論は確立されていない。そこで、本研究ではシミュレーションにおいてさまざまなパラメータに応じた投擲フォームを導出・比較することで、さまざまな投擲フォームの戦略を検討・考察・議論することを目的とする。\\
        研究室が散らかっている(\figref{photo.jpg}参照)ので、片付けるロボットが欲
        しい。

        この図(\figref{fig1.eps})はなんだろう?

        \secref{intro}ではほげほげ。

        こういう研究\cite{Ikuo:doctor}もありました。

        ああいう研究\cite{Hondo:JRSJ2011}もありました。

        bibファイルでは、著者名(author=)は、
        「苗字 名前 and 苗字 名前 and 苗字 名前」
        のようにするんですよ\cite{Mizuuchi:RSJ2015-baneoid}。
        間は全部半角スペースですよ。

        \section{本論文の構成}
        本論文では、全6章から構成される。以下に、各章の概要について述べる。
        \begin{itemize}
          \item 第1章(本章)では、研究の背景と目的について述べた。
          \item 第2章「強化学習を用いた投擲フォーム導出」では、本研究の学習手法として用いた強化学習について述べる。
          \item 第3章「剛体1リンクモデルによるリンク速度最適化シミュレーション」では、投擲フォームでの強化学習に先立ち、自作したシミュレータと手法の有用性について述べる。
          \item 第4章「2次元剛体2リンクによるパラメータに応じた投擲フォームの導出・比較と戦略の考察」では、腕に見立てた2次元剛体2リンクモデルを用いて投擲物の重さに応じた投擲フォーム戦略、腕の長さに応じた投擲フォーム戦略の結果・考察について述べる。
          \item 第5章「3次元剛体2リンクによるパラメータに応じた投擲フォームの導出・比較と戦略の考察」では、腕のモデルを3次元に拡張し、投擲物の重さに応じた投擲フォーム戦略、腕の長さに応じた投擲フォーム戦略の結果・考察について述べる。
          \item 第6章「結論および今後の展望」では、本研究で得られた結論および今後の展望について述べる。
        \end{itemize}
