\chapter[長いタイトルを改行する場合はこのようにしましょう(見出し用)]%
        {長いタイトルを改行する場合は\\このようにしましょう}

        \figb{photo.jpg}{width=0.75\hsize}{ある日の研究室}%

        \fig{fig1.eps}{width=.9\hsize}{何かの図}%

        \section{研究の背景と目的}
        \seclabel{intro}

        研究室が散らかっている(\figref{photo.jpg}参照)ので、片付けるロボットが欲
        しい。

        この図(\figref{fig1.eps})はなんだろう?

        \secref{intro}ではほげほげ。

        こういう研究\cite{Ikuo:doctor}もありました。

        ああいう研究\cite{Hondo:JRSJ2011}もありました。

        bibファイルでは、著者名(author=)は、
        「苗字 名前 and 苗字 名前 and 苗字 名前」
        のようにするんですよ\cite{Mizuuchi:RSJ2015-baneoid}。
        間は全部半角スペースですよ。

        \section{従来研究}

        \begin{table}[tb]
          \tablabel{hogehoge}
          \begin{center}
            \caption{試しに作った表}
            \begin{tabular}{l|c|r|r}
              \hline
              項目 & 数値 & コメント & 備考 \\
              \hline
              a & 10.0 & こめんとしがたい & どうすべ?\\
              b & 20.0 & こめんとしがたい & どうすべ?\\
              c & -100.0 & こめんとしがたい & どうすべ?\\
              \hline
            \end{tabular}
          \end{center}
        \end{table}

        \tabref{hogehoge}に、何かの表を示す。

        \section{本論文の構成}
